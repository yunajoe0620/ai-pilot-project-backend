
        \documentclass[fleqn]{article}      
        \usepackage{amsmath}
        \usepackage{fontspec}
        \usepackage{kotex} % 한국어 지원  

        \begin{document}      
        \noindent \textbf{쉬운 문제} \\

1. 다음 행렬 \( A = \begin{pmatrix} 1 & 2 \\ 3 & 4 \end{pmatrix} \)의 행렬식은 얼마인가? \\ 
\\\\\\

2. 다음 행렬 \( B = \begin{pmatrix} 0 & 5 \\ 1 & 2 \end{pmatrix} \)에 대해 \( B \)의 전치 행렬은 무엇인가? \\ 
\\\\\\

3. 다음 중 \( 2 \times 2 \) 행렬의 개수를 고르시오: \\
   (A) 3 \\ 
   (B) 4 \\ 
   (C) 5 \\ 
   (D) 6 \\ 
\\\\\\

4. 다음 행렬 \( C = \begin{pmatrix} 2 & 3 \\ 5 & 7 \end{pmatrix} \)의 대각합은 얼마인가? \\ 
\\\\\\

5. 주어진 행렬 \( D = \begin{pmatrix} 1 & 0 \\ 0 & 1 \end{pmatrix} \)이 단위 행렬임을 설명하시오. \\ 
\\\\\\

\noindent \textbf{보통 문제} \\

1. 두 행렬 \( A = \begin{pmatrix} 1 & -1 \\ 2 & 0 \end{pmatrix} \)과 \( B = \begin{pmatrix} 3 & 4 \\ -1 & 2 \end{pmatrix} \)의 곱 \( AB \)는 무엇인가? \\ 
\\\\\\

2.  행렬 \( E = \begin{pmatrix} 4 & 2 \\ 2 & 3 \end{pmatrix} \)의 고유값을 구하시오. \\ 
\\\\\\

3. 행렬 \( F = \begin{pmatrix} 1 & 2 & 3 \\ 0 & 1 & 4 \\ 0 & 0 & 1 \end{pmatrix} \)의 역행렬을 구하시오. \\ 
\\\\\\

4. 2x2 행렬 \( G = \begin{pmatrix} a & b \\ c & d \end{pmatrix} \)의 행렬식이 0이 되도록 하는 조건은 무엇인가? \\ 
\\\\\\

5. \( H = \begin{pmatrix} 2 & 0 \\ 0 & 3 \end{pmatrix} \)에서 \( H \)의 고유값을 구하시오. \\ 
\\\\\\

\noindent \textbf{어려운 문제} \\

1. 3x3 행렬 \( I = \begin{pmatrix} 1 & -1 & 0 \\ 0 & 1 & -2 \\ 3 & 1 & 1 \end{pmatrix} \)의 역행렬을 구하시오. \\ 
\\\\\\

2. 주어진 행렬 \( J = \begin{pmatrix} 3 & 1 & 2 \\ 1 & 0 & 1 \\ 2 & 3 & 2 \end{pmatrix} \)의 고유값을 구하시오. \\ 
\\\\\\

3. 두 행렬 \( K = \begin{pmatrix} 1 & 1 \\ 0 & 1 \end{pmatrix} \)과 \( L = \begin{pmatrix} 1 & 2 \\ 3 & 4 \end{pmatrix} \)의 곱에서 \( KL \)에 대한 표현식을 구하시오. \\ 
\\\\\\

4. 다음 문제에서 행렬 \( M = \begin{pmatrix} 2 & 3 \\ 4 & 5 \end{pmatrix} \)의 고유벡터를 구하시오. \\ 
\\\\\\

5. 주어진 행렬 \( N = \begin{pmatrix} 1 & 2 \\ 3 & 4 \\ 5 & 6 \end{pmatrix} \)로부터 전치 행렬을 구하고, 그 행렬의 행렬식을 계산하시오. \\ 
\\\\\\

\noindent \textbf{정답} \\

1. \( -2 \) \\ 
\\\\\\

2. \( \begin{pmatrix} 0 & 1 \\ 5 & 2 \end{pmatrix} \) \\ 
\\\\\\

3. (C) 5 \\ 
\\\\\\

4. \( 9 \) \\ 
\\\\\\

5. 단위 행렬은 주대각선 원소가 1이고 나머지가 0인 행렬입니다. \\ 
\\\\\\

1. \( \begin{pmatrix} 1 & 1 \\ 5 & 2 \end{pmatrix} \) \\ 
\\\\\\

2. \( \lambda^2 - 7\lambda + 10 = 0 \)으로부터 \( \lambda = 2, 5 \) \\ 
\\\\\\

3. \( \begin{pmatrix} 1 & 3 & 5 \\ 0 & 1 & 4 \\ 0 & 0 & 1 \end{pmatrix} \) \\ 
\\\\\\

4. \( ad - bc = 0 \) \\ 
\\\\\\

5. \( 2, 3 \) \\ 
\\\\\\

1. \( \begin{pmatrix} -1 & 2 & 1 \\ 1 & 1 & -1 \\ -1 & -3 & 1 \end{pmatrix} \) \\ 
\\\\\\

2. \( \lambda = 0, 6 \) \\ 
\\\\\\

3. \( \begin{pmatrix} 4 & 6 \\ 3 & 2 \end{pmatrix} \) \\ 
\\\\\\

4. \( \begin{pmatrix} 1 & 3 & 5 \\ 4 & 8 & 10 \\ 2 & 4 & 6 \end{pmatrix} \)의 행렬식 = 0 \\ 
\\\\\\      
        \end{document} 
    